

\usepackage{acronym}
\usepackage{bm}% bold math
\usepackage{graphicx} % For including graphics
\usepackage{subcaption} % For subfigure support


\usepackage{placeins}
\usepackage[table]{xcolor}
\usepackage{tikz}
\usepackage{amsmath}
\usepackage{amssymb}
\usepackage{array}
\usepackage{layouts}
\usepackage{xspace}

% Automatically add \centering to all figure environments
\usepackage{etoolbox}
\makeatletter
\g@addto@macro\@floatboxreset\centering
\makeatother


\graphicspath{{figures}}
\captionsetup[figure]{justification=centerlast} 

\acrodef{2C-2D}[2C\allowbreak-2D]{two–component – two–dimensional}
\acrodef{APG}[APG]{adverse pressure gradient}
\acrodef{CDF}[CDF]{cumulative distribution function}
\acrodef{DNS}[DNS]{direct numerical simulation}
\acrodef{FOV}[FOV]{field of view}
\acrodef{FPG}[FPG]{favourable pressure gradient}
\acrodef{HSR}[HSR]{high–spatial–resolution}
\acrodef{HWA}[HWA]{hot–wire anemometry}
\acrodef{LIF}[LIF]{laser–induced fluorescence}
\acrodef{PDF}[PDF]{probability density function}
\acrodef{PIV}[PIV]{particle image velocimetry}
\acrodef{ROI}[ROI]{region of interest}
\acrodef{SS-APG-TBL}[SS\allowbreak–APG\allowbreak–TBL]{self–similar adverse pressure gradient turbulent boundary layer}
\acrodef{TBL}[TBL]{turbulent boundary layer}
\acrodef{TKE}[TKE]{turbulent kinetic energy}
\acrodef{TNTI}[TNTI]{turbulent/non–turbulent interface}
\acrodef{UMZ}[UMZ]{uniform momentum zone}
\acrodef{ZPG}[ZPG]{zero pressure gradient}



\newcommand{\fref}[1]{figure \ref{#1}}
\newcommand{\Fref}[1]{Figure \ref{#1}}
\newcommand{\tref}[1]{table \ref{#1}}
\newcommand{\Tref}[1]{Table \ref{#1}}
\newcommand{\eref}[1]{equation \ref{#1}}
\newcommand{\Eref}[1]{Equation \ref{#1}}
\newcommand{\sref}[1]{section \ref{#1}}
\newcommand{\Sref}[1]{Section \ref{#1}}

\definecolor{plt1}{RGB}{31, 119, 180}
\definecolor{plt2}{RGB}{255, 127, 14}
\definecolor{plt3}{RGB}{44, 160, 44}
\definecolor{plt4}{RGB}{214, 39, 40}
\definecolor{plt5}{RGB}{148, 103, 189}
\definecolor{plt6}{RGB}{140, 86, 75}
\definecolor{red}{RGB}{255, 0, 0}
\definecolor{blue}{RGB}{0, 0, 255}
\definecolor{black}{RGB}{0,0,0}
\definecolor{laser532Green}{RGB}{101, 255, 0}
\definecolor{laser457Blue}{RGB}{0, 108, 255}
\definecolor{laser633Red}{RGB}{255, 66, 0}


\DeclareRobustCommand\dashed{\tikz[baseline=-0.6ex] \draw[thick,dash pattern=on 3pt off 3pt] (0,0)--(15pt,0);\xspace}
\DeclareRobustCommand\dotdash{\tikz[baseline=-0.6ex]\draw[thick, dash pattern=on 1pt off 2pt on 4pt off 2pt] (0,0)--(15pt,0);\xspace}


\DeclareRobustCommand\line{\tikz[baseline=-0.6ex]\draw[thick] (0,0)--(15pt,0);\xspace}
\DeclareRobustCommand\solidCircle[1]{\tikz[baseline=-0.6ex]\draw[#1,fill=#1,radius=0.25em] (0,0) circle ;}
\DeclareRobustCommand\lineCircle[1]{\tikz[baseline=-0.6ex] \draw[#1,fill=#1,thick] 
    (-0.75em,0) -- (0,0) 
    (0,0) circle[radius=0.25em] 
    (0,0) -- (0.75em,0);\xspace
}
\DeclareRobustCommand\lineSquare[1]{\tikz[baseline=-0.6ex] \draw[#1,fill=#1,thick] 
    (-0.75em,0) -- (0,0) 
    (-0.25em,-0.25em) rectangle (0.25em,0.25em)  
    (0,0) -- (0.75em,0);\xspace
}
\DeclareRobustCommand\lineTriangle[1]{\tikz[baseline=-0.6ex] \draw[#1,fill=#1,thick] 
    (-0.75em,0) -- (0,0) 
    (-0.25em,-0.2165em) -- (0.25em,-0.2165em) -- (0em,0.2887em) -- cycle  
    (0,0) -- (0.75em,0);\xspace
}
\newcommand{\lineBreakCell}[2][c]{%
  \begin{tabular}[#1]{@{}c@{}}#2\end{tabular}}
  
